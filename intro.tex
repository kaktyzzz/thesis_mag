\intro

Каждый пользователь глобальной сети «Интернет» ежедневно сталкивается с
необходимостью входа на какой-либо ресурс, зарегистрировав себя в системе. Этот
процесс необходим для:
\begin{itemize}
  \item обеспечения конфиденциальности важных данных пользователя, который
  хранит их на данном ресурсе;
\item разграничения прав доступа к этому ресурсу.
\end{itemize}

Процесс регистрации пользователя в системе состоит из трех взаимосвязанных,
выполняемых последовательно процедур: идентификации, аутентификации и
авторизации.

Идентификация --- это процедура распознавания субъекта по его идентификатору. В
процессе регистрации субъект предъявляет системе свой идентификатор и она
проверяет его наличие в своей базе данных. Субъекты с известными системе
идентификаторами считаются легальными (законными), остальные субъекты относятся
к нелегальным.

Аутентификация --- процедура проверки подлинности субъекта, позволяющая
достоверно убедиться в том, что субъект, предъявивший свой идентификатор, на
самом деле является именно тем субъектом, идентификатор которого он использует.
Для этого он должен подтвердить факт обладания некоторой информацией, которая
может быть доступна только ему одному (пароль, ключ и т.п.).

Авторизация --- процедура предоставления субъекту определенных прав доступа к
ресурсам системы после прохождения им процедуры аутентификации. Для каждого
субъекта в системе определяется набор прав, которые он может использовать при
обращении к ее ресурсам.

Для того чтобы обеспечить управление и контроль над данными процедурами,
дополнительно используются процессы администрирования и аудита.

Администрирование --- процесс управления доступом субъектов к ресурсам системы.
Данный процесс включает в себя:
\begin{itemize}
  \item создание идентификатора субъекта (создание учетной записи пользователя) в
системе;
\item управление данными субъекта, используемыми для его аутентификации
(смена пароля, издание сертификата и т.п.);
\item управление правами доступа
субъекта к ресурсам системы.
\end{itemize}

Аудит --- процесс контроля (мониторинга) доступа субъектов к ресурсам системы,
включающий протоколирование действий субъектов при их доступе к ресурсам системы
в целях обеспечения возможности обнаружения несанкционированных действий.
В общем случае, речь идет о пяти основных процедурах процесса
предоставления доступа к информации.

Под системой доступа понимается набор методов и способов, а так же
соответствующий программно-аппаратный комплекс, обеспечивающие процесс
предоставления доступа к конфиденциальной информации.

Идентификационная информация --- набор уникальных параметров, в том числе
секретных, позволяющих пользователю получить доступ к системе.
%
% Используемые далее команды определяются в файле common.tex.
%

% Актуальность работы
\actualitysection
\actualitytext

%Объект исследования
\objectsection
\objecttext

%Предмет исследования
\subjectsection
\subjecttext

% Цель диссертационной работы
\objectivesection
\objectivetext

% Научная новизна
\noveltysection
\noveltytext

% Практическая ценность
\valuesection
\valuetext

% Результаты и положения, выносимые на защиту
\resultssection
\resultstext

% Апробация работы
\approbationsection
\approbationtext

% Публикации
\pubsection
\pubtext

% Личный вклад автора
\contribsection
\contribtext

% Структура и объем диссертации
\structsection
\structtext