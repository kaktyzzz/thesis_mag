\section{Сравнительный анализ систем доступа}

\begin{table}[pH]
  \centering
  \small
  \begin{tabular}{|p{2.5cm}|p{3.5cm}|p{2.5cm}|p{2.75cm}|p{3.25cm}|}
    \hline
    
	\textbf{Метод доступа} & \textbf{Безопасность} & \textbf{Сложность в
	использовании} & \textbf{Сложность внедрения} & \textbf{Другие недостатки} \\
	\hline
	
\textit{На основе ввода логина и пароля} & -- уязвимость в отношении вредоносных
программ; 

-- пароль можно подсмотреть. & необходим ручной ввод логина и пароля
& относительно просто в реализации и внедрении & дополнительная уязвимость при
восстановлении пароля  \\ \hline

\textit{На основе биометрических показателей человека} & изменение структуры
человека приводит к отказу в работе системы & простота в использовании & --
существует сложность внедрения; -- высокая стоимость оборудования & существует процент
  сбоев при проверке пользователя \\ \hline
  
\textit{С помощью генератора одноразовых паролей} & высокая безопасность за счёт
 короткого промежутка времени жизни пароля & необходим ручной ввод пароля &
 существует сложность внедрения & для работы системы необходима синхронизация
 времени, что не просто осуществить\\ \hline
  
 \textit{C помощью смарт-карты} & -- высокая безопасность;
 
-- использование двухфакторной авторизации & простота в использовании &
существует сложность внедрения & -- для работы устройства необходим специальный
считыватель;

-- уязвимость при вводе PIN кода \\ \hline
  
  \textit{С помощью usb-ключа} & уязвимость в отношении вредоносных программ &
  простота в использовании & относительная простота  реализации и внедрения & \\
  \hline
  \end{tabular}
  \caption{Сравнение существующих систем доступа к информационным ресурсам}
  \label{tab:3}
\end{table}

Для проведения сравнительного анализа систем доступа необходимо выбрать критерий
или ряд критериев, по которым будут оцениваться рассмотренные ранее методы и
программно-технические средства.

В качестве главного критерия оценки предлагается взять степень защищенности
данных пользователя от несанкционированного проникновения. Немаловажным
критерием будет наличие многофакторной аутентификации с применением методов
шифрования.

Еще одним критерием оценки систем доступа может выступать сложность внедрения и
использования. В частности речь идёт о способе внедрения, позволяющего не
вмешиваться в существующее функционирование порталов. В плане использования
можно выдвинуть оценку -- устойчивость к возможности проявления халатностей
пользователя различного рода, а так же количество простейших операций,
выполняемых пользователем для осуществления доступа.

Применительно к корпоративным информационным системам стоит выделить критерий,
определяющий возможность использования выбранного метода доступа для
осуществления электронного документооборота с помощью ЭЦП.

В ходе исследования были рассмотрены следующие методы регистрации пользователя в
системе:
\begin{itemize}
  \item осуществление доступа на основе биометрических показателей человека;
  \item осуществление доступа с помощью генератора одноразовых паролей;
  \item осуществление доступа с помощью смарт-карты;
  \item осуществление доступа с помощью USB-ключа.
\end{itemize}

Можно сказать, что ни одна из рассмотренных систем доступа не лишена
недостатков. Это подтверждает таблица~\ref{tab:3}.



Ввиду обнаруженных недостатков в существующих методах аутентификации априори
возникает необходимость их устранения. Они ослабляют и подрывают весь комплекс
защитных мер личных данных. Сложность внедрения и использования повышает роль
человеческого фактора в осуществлении механизмов безопасности. Пользователи
находят трудную реализацию утомительной и перестают использовать весь спектр
защитных возможностей, намеренно делая систему уязвимой. Слабая
криптоустойчивость некоторых существующих механизмов безопасности изначально
снижает уровень защищенности системы.
