% Общие поля титульного листа диссертации и автореферата
\institution{\small{МИНИСТЕРСТВО ОБРАЗОВАНИЯ И НАУКИ РОССИЙСКОЙ ФЕДЕРАЦИИ
ФЕДЕРАЛЬНОЕ ГОСУДАРСТВЕННОЕ БЮДЖЕТНОЕ ОБРАЗОВАТЕЛЬНОЕ УЧРЕЖДЕНИЕ ВЫСШЕГО
ПРОФЕССИОНАЛЬНОГО ОБРАЗОВАНИЯ <<ГОСУДАРСТВЕННЫЙ
УНИВЕРСИТЕТ --- УЧЕБНО-НАУЧНО-ПРОИЗВОДСТВЕННЫЙ КОМПЛЕКС>>}}

\topic{Разработка подсистемы аутентификации пользователей в сети корпоративных
информационных порталов с применением портативного цифрового ключа доступа}

\author{Силаев Павел Павлович}

\specnum{080800.68}
\spec{Системы корпоративного управления}
%\specsndnum{01.04.07}
%\specsnd{Физика конденсированного состояния}

\sa{Лазарев Сергей Александрович}
\sastatus{к.~э.~н., доцент.}
%\sasnd{ФИО второго руководителя}
%\sasndstatus{к.~ф.-м.~н., проф.}

%\scon{ФИО консультанта}
%\sconstatus{д.~ф.-м.~н., проф.}

\city{Орел}
\date{\number\year}

% Общие разделы автореферата и диссертации
\mkcommonsect{actuality}{Актуальность работы.}{%
В современном мире к корпоративным
информационным системам (далее КИС) предъявляют высочайшие требования. В первую
очередь это касается функциональных требований и, как следствие, требований безопасности.
Одним из важнейших элементов КИС является доступ к корпоративным
информационным порталам, к которым всегда предъявлялись особые требования по
обеспечению безопасности.

С развитием информационных технологий резко возросло количество
несанкционированных проникновений. Это затрагивает вопрос о разработке
единого, универсального и высокозащищенного механизма доступа к
web-порталам, что, в свою очередь, позволит развивать корпоративные
информационные системы, системы электронного документооборота, а так же
способствует созданию информационных ассоциаций.~\cite{accociate}
Поэтому решение данной проблемы является сейчас одной из самых актуальных
задач.}

\mkcommonsect{object}{Объектом исследования}{%
является совокупность методик, моделей, алгоритмов и аппаратных средств
распределённой системы доступа к сети корпоративных порталов.}

\mkcommonsect{subject}{Предметом исследования}{% выступают процессы
осуществления доступа к порталам по средствам распределённой системы с помощью
портативных носителей идентификационной информации.}

\mkcommonsect{objective}{Цель диссертационной работы}{%
состоит в разработке подсистемы аутентификации
пользователей в сети корпоративных
информационных порталов с применением
портативного цифрового ключа доступа

Для достижения поставленной цели были решены следующие задачи:
\begin{enumerate}
  \item Анализ способов доступа к сети корпоративных порталов;
  \item Создание формализованных
методов и моделей функционирования системы доступа к сети корпоративных
информационных порталов;
\item Формирование технологических и алгоритмических решений
по построению системы доступа к корпоративным информационным порталам;
\item Реализация
программно-аппаратного средства доступа к сети корпоративных порталов с помощью
портативного цифрового ключа доступа.
\end{enumerate}
}

\mkcommonsect{novelty}{Научная новизна}{%
исследования заключается в разрабатываемых:
\begin{itemize}
  \item технологии доступа к сети порталов; 
  \item модели функционирования сети корпоративных информационных  порталов.
\end{itemize} 
}

\mkcommonsect{value}{Практическая значимость}{%
Результаты, изложенные в диссертации, могут быть использованы совместно с
распределенной системой управления доступом для сети
порталов.~\cite{conf_itnop_lsa_concept}}

\mkcommonsect{results}{%
На защиту выносятся следующие основные результаты и положения:}{%
Текст о результатах.
}

\mkcommonsect{approbation}{Апробация работы}{%
Основные результаты диссертации докладывались на следующих конференциях: 
}

\mkcommonsect{pub}{Публикации.}{%
Материалы диссертации опубликованы в $5$ печатных работах, из них $2$
статьи в рецензируемых журналах вхожящих в перечень ВАК, $2$ статьи в
сборниках трудов конференций.
}

\mkcommonsect{contrib}{Личный вклад автора}{% 
Содержание диссертации и основные
положения, выносимые на защиту, отражают персональный вклад автора в
опубликованные работы.
Подготовка к публикации полученных результатов проводилась совместно с
соавторами, причем вклад диссертанта был определяющим. Все представленные в
диссертации результаты получены лично автором.
}

\mkcommonsect{struct}{Структура и объем диссертации}{%
Диссертация состоит из введения,  $4ех$ глав, заключения и библиографии.
Общий объем диссертации $p_1$ страниц, из них $p_2$ страницы текста, включая $p_3$ рисунков.
Библиография включает $b_1$ наименований на $p_4$ страницах.
}