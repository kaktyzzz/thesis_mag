\section{Тестирование системы}

На завершающей стадии разработки системы было проведено комплексное
тестирование. Для поведения тестовых испытаний была разработана формальная
методика.

\textbf{Программа и методика проведения испытаний системы} на этапе опытного
функционирования предназначена для установления данных, обеспечивающих получение
и проверку проектных решений, выявление причин сбоев, определение качества
работ, показателей качества функционирования системы, проверку соответствия
системы требованиям техники безопасности, продолжительность и режим испытаний.

Согласно РД 50-34.698-90 «Автоматизированные системы требования к содержанию
документов», программа испытаний содержит следующие разделы:
\begin{itemize}
  \item объект испытаний;
  \item цель испытаний;
  \item общие положения;
  \item объем испытаний;
  \item условия и порядок проведения испытаний;
  \item материально-техническое обеспечение
испытаний;
  \item отчетность;
  \item методика испытаний.
\end{itemize}

\textbf{Объект испытаний}

Объектом испытаний является подсистема аутентификации
пользователей в сети корпоративных информационных порталов с применением
портативного цифрового ключа доступа (далее система).

Испытания проводятся для всех следующих функций системы:
\begin{itemize}
  \item аутентификация пользователя в корпоративном информационном портале с
использованием портативного цифрового ключа доступа;
  \item управление учетными
записями пользователей;
  \item создание и установка секретных ключей на портативное
цифровое устройство доступа;
  \item деактивация ключей по истечению срока службы или
по требованию;
  \item размещение файлов на портале;
  \item подтверждение подлинности
документов (файлов) с использованием портативного цифрового ключа доступа.
\end{itemize}

\textbf{Цель испытаний}

Целью проведения испытаний является:
\begin{itemize}
  \item проверка взаимодействия подсистем;
  \item проверка работоспособности системы под
нагрузкой;
  \item проверка соответствия системы заявленным функциональным
требованиям;
  \item проверка соответствия системы требованиям приведенным техническом
задании.
\end{itemize}

\textbf{Общие положения}

Настоящая программа и методика испытаний разработана в соответствии со
следующими документами:
\begin{itemize}
  \item ГОСТ 34.603-92 Виды испытаний автоматизированных систем;
  \item РД 50-34.698-90
Автоматизированные системы требования к содержанию документов;
  \item ГОСТ 19.301-79
Программа и методика испытаний. Требования к содержанию и оформлению;
  \item РД 50-34.698-90 Методические указания информационная технология комплекс
  стандартов и руководящих документов на автоматизированные системы автоматизированные
системы требования к содержанию документов;
  \item техническое задание.
\end{itemize} 

Ниже приведен перечень программной документации, предъявляемой для
использования:
\begin{itemize}
  \item Общее описание системы;
  \item Руководство пользователя;
  \item Руководство
администратора;
  \item Описание программ;
  \item Тексты программ.
\end{itemize} 

\textbf{Объем испытаний} 

В таблице \ref{tab:4-2} представлен перечень требований накладываемых на систему.

\begin{longtable}[h!]{|p{0.5cm}|p{2.6cm}|p{4cm}|p{3cm}|p{5cm}|}
\caption[dfgdshfdfh]{Перечень требований}  \label{tab:4-2} \\ \hline
\small
%\textbf{\No} & \textbf{Объект испытаний} & \textbf{Наименование испытания} &
\textbf{Вид испытания} & \textbf{Оцениваемые характеристики} \\ \hline

1 & Подсистема аутентификации &	Проверка базовой функциональности на уровне
пользователя & автономное & 
\vspace{-22pt}
\begin{list}{-}{\topsep=0pt\parskip=0pt\partopsep=0pt}
  \item Предоставление доступа к ресурсам портала;
  \item Вход на портал под заданным именем;
  \item Поддержание сеанса с пользователем.
\end{list}	\\ \hline

2 &	Подсистема обмена файлами и подтверждения их подлинности & Проверка базовой
функциональности на уровне пользователя & автономное
\vspace{-22pt}
\begin{list}{-}{\topsep=0pt\parskip=0pt\partopsep=0pt}
  \item Размещение файлов на
сервере;
  \item Утверждение и проверка подлинности для файлов.
\end{list} \\ \hline

3 & Подсистема управления пользователями и секретными ключами & Проверка
управления пользователями и секретными ключами & автономное	& 
\vspace{-22pt}
\begin{list}{-}{\topsep=0pt\parskip=0pt\partopsep=0pt}
  \item Мониторинг
активности пользователей;
  \item Модификация учетных записей;
  \item Создание и удаление учетных записей пользователей;
  \item Создание и установка секретных ключей на цифровое устройство;
  \item Деактивация ключей.
\end{list}
1.  2.  3. 
4. 
5. 

\end{longtable}