\documentclass[autoref,subf,href
%,facsimile
%,fixint=false
%,times
%,classified
]{disser}

\usepackage[a4paper,nohead,includefoot,mag=1000,
            margin=2cm,footskip=1cm]{geometry}
\usepackage[T2A]{fontenc}
\usepackage[cp1251]{inputenc}
\usepackage[english,russian]{babel}
\usepackage{tabularx}
\ifpdf\usepackage{epstopdf}\fi

% Поддержка нескольких списков литературы в одном документе
\usepackage{multibib}
% Создание команд для цитирования собственных работ диссертанта
% в отдельном разделе. В данном случае ссылка будет иметь вид \citemy{...}.
\newcites{my}{Список публикаций}

% Путь к файлам с иллюстрациями
\graphicspath{{fig/}}

\begin{document}
% Включение файла с общим текстом диссертации и автореферата
% (текст титульного листа и характеристика работы).
% Общие поля титульного листа диссертации и автореферата
\institution{\small{МИНИСТЕРСТВО ОБРАЗОВАНИЯ И НАУКИ РОССИЙСКОЙ ФЕДЕРАЦИИ
ФЕДЕРАЛЬНОЕ ГОСУДАРСТВЕННОЕ БЮДЖЕТНОЕ ОБРАЗОВАТЕЛЬНОЕ УЧРЕЖДЕНИЕ ВЫСШЕГО
ПРОФЕССИОНАЛЬНОГО ОБРАЗОВАНИЯ <<ГОСУДАРСТВЕННЫЙ
УНИВЕРСИТЕТ --- УЧЕБНО-НАУЧНО-ПРОИЗВОДСТВЕННЫЙ КОМПЛЕКС>>}}

\topic{Разработка подсистемы аутентификации пользователей в сети корпоративных
информационных порталов с применением портативного цифрового ключа доступа}

\author{Силаев Павел Павлович}

\specnum{080800.68}
\spec{Системы корпоративного управления}
%\specsndnum{01.04.07}
%\specsnd{Физика конденсированного состояния}

\sa{Лазарев Сергей Александрович}
\sastatus{к.~э.~н., доцент.}
%\sasnd{ФИО второго руководителя}
%\sasndstatus{к.~ф.-м.~н., проф.}

%\scon{ФИО консультанта}
%\sconstatus{д.~ф.-м.~н., проф.}

\city{Орел}
\date{\number\year}

% Общие разделы автореферата и диссертации
\mkcommonsect{actuality}{Актуальность работы.}{%
В современном мире к корпоративным
информационным системам (далее КИС) предъявляют высочайшие требования. В первую
очередь это касается функциональных требований и, как следствие, требований безопасности.
Одним из важнейших элементов КИС является доступ к корпоративным
информационным порталам, к которым всегда предъявлялись особые требования по
обеспечению безопасности.

С развитием информационных технологий резко возросло количество
несанкционированных проникновений. Это затрагивает вопрос о разработке
единого, универсального и высокозащищенного механизма доступа к
web-порталам, что, в свою очередь, позволит развивать корпоративные
информационные системы, системы электронного документооборота, а так же
способствует созданию информационных ассоциаций.~\cite{accociate}
Поэтому решение данной проблемы является сейчас одной из самых актуальных
задач.}

\mkcommonsect{object}{Объектом исследования}{%
является совокупность методик, моделей, алгоритмов и аппаратных средств
распределённой системы доступа к сети корпоративных порталов.}

\mkcommonsect{subject}{Предметом исследования}{% выступают процессы
осуществления доступа к порталам по средствам распределённой системы с помощью
портативных носителей идентификационной информации.}

\mkcommonsect{objective}{Цель диссертационной работы}{%
состоит в разработке подсистемы аутентификации
пользователей в сети корпоративных
информационных порталов с применением
портативного цифрового ключа доступа

Для достижения поставленной цели были решены следующие задачи:
\begin{enumerate}
  \item Анализ способов доступа к сети корпоративных порталов;
  \item Создание формализованных
методов и моделей функционирования системы доступа к сети корпоративных
информационных порталов;
\item Формирование технологических и алгоритмических решений
по построению системы доступа к корпоративным информационным порталам;
\item Реализация
программно-аппаратного средства доступа к сети корпоративных порталов с помощью
портативного цифрового ключа доступа.
\end{enumerate}
}

\mkcommonsect{novelty}{Научная новизна}{%
исследования заключается в разрабатываемых:
\begin{itemize}
  \item технологии доступа к сети порталов; 
  \item модели функционирования сети корпоративных информационных  порталов.
\end{itemize} 
}

\mkcommonsect{value}{Практическая значимость}{%
Результаты, изложенные в диссертации, могут быть использованы совместно с
распределенной системой управления доступом для сети
порталов.~\cite{conf_itnop_lsa_concept}}

\mkcommonsect{results}{%
На защиту выносятся следующие основные результаты и положения:}{%
Текст о результатах.
}

\mkcommonsect{approbation}{Апробация работы}{%
Основные результаты диссертации докладывались на следующих конференциях: 
}

\mkcommonsect{pub}{Публикации.}{%
Материалы диссертации опубликованы в $5$ печатных работах, из них $2$
статьи в рецензируемых журналах вхожящих в перечень ВАК, $2$ статьи в
сборниках трудов конференций.
}

\mkcommonsect{contrib}{Личный вклад автора}{% 
Содержание диссертации и основные
положения, выносимые на защиту, отражают персональный вклад автора в
опубликованные работы.
Подготовка к публикации полученных результатов проводилась совместно с
соавторами, причем вклад диссертанта был определяющим. Все представленные в
диссертации результаты получены лично автором.
}

\mkcommonsect{struct}{Структура и объем диссертации}{%
Диссертация состоит из введения,  $4ех$ глав, заключения и библиографии.
Общий объем диссертации $p_1$ страниц, из них $p_2$ страницы текста, включая $p_3$ рисунков.
Библиография включает $b_1$ наименований на $p_4$ страницах.
}

% номер копии для грифа секретности
%\copynum{1}
% класс доступа
%\classlabel{Для служебного пользования}

\title{АВТОРЕФЕРАТ\\
диссертации на соискание ученой степени\\
кандидата физико-математических наук}

\maketitle

% Внутренняя сторона обложки
\noindent
\begin{center}
Работа выполнена в \emph{название организации}.
\end{center}
\vskip1ex
\begin{tabularx}{\linewidth}{lp{1cm}X}
Научный руководитель:  & & \emph{ученая степень}, \\
                       & & \emph{ученое звание}, \\
                       & & \emph{фамилия имя отчество}
\\
Официальные оппоненты: & & \emph{ученая степень}, \\
                       & & \emph{ученое звание}, \\
                       & & \emph{фамилия имя отчество}\\
                       & & \emph{ученая степень}, \\
                       & & \emph{ученое звание}, \\
                       & & \emph{фамилия имя отчество}
\\
Ведущая организация:   & & \emph{название организации}\\
\end{tabularx}

\vskip2ex\noindent
Защита состоится \datefield{} в \rule[0pt]{1cm}{0.5pt} часов
на заседании диссертационного совета \emph{шифр совета} при \emph{название
организации, при которой создан совет}, расположенном по адресу:
\emph{адрес}

\vskip1ex\noindent
С диссертацией можно ознакомиться в библиотеке
\emph{название организации}.

\vskip1ex\noindent
Автореферат разослан \datefield{}

\vskip2ex\noindent
Отзывы и замечания по автореферату в двух экземплярах, заверенные
печатью, просьба высылать по вышеуказанному адресу на имя ученого секретаря
диссертационного совета.

\vfill\noindent
Ученый секретарь\\
диссертационного совета,\\
\emph{ученая степень}, \emph{ученое звание}%
\hfill
\makeatletter
% вставка файла, содержащего факсимиле ученого секретаря
%\ifDis@facsimile
%  \raisebox{-4pt}{\includegraphics[width=3cm]{sec-facsimile}}\hfill
%\fi%
\makeatother%
\emph{фамилия и. о.}

\clearpage

\nsection{Общая характеристика работы}

% Актуальность работы
\actualitysection
\actualitytext

% Цель диссертационной работы
\objectivesection
\objectivetext

% Научная новизна
\noveltysection
\noveltytext

% Практическая значимость
\valuesection
\valuetext

% Результаты и положения, выносимые на защиту
\resultssection
\resultstext

% Апробация работы
\approbationsection
\approbationtext

% Публикации
\pubsection
\pubtext

% Личный вклад автора
\contribsection
\contribtext

% Структура и объем диссертации
\structsection
\structtext

\nsection{Содержание работы}

\textbf{Во Введении} обоснована актуальность диссертационной работы,
сформулирована цель и аргументирована научная новизна исследований, показана
практическая значимость полученных результатов, представлены выносимые на
защиту научные положения.

\textbf{В первой главе} ...

Содержание первой главы.

Результаты первой главы опубликованы в
работе~\citemy{Ivanov_1999_Journal_17_173}.

\textbf{Во второй главе} ...

Содержание второй главы.

Результаты второй главы опубликованы в
работе~\citemy{Petrov_2001_Journal_23_12321}.

\textbf{В третьей главе} ...

Содержание третьей главы.

Результаты третьей главы опубликованы в
работе~\citemy{Sidorov_2002_Journal_32_1531}.

\textbf{В Заключении}

% ----------------------------------------------------------------
\renewcommand\bibsection{\nsection{Список публикаций}}

% Префикс номеров ссылок на работы соискателя
\def\BibPrefix{A}
\bibliographystylemy{gost705s}
\bibliographymy{main}

\renewcommand\bibsection{\nsection{Цитированная литература}}

\def\BibPrefix{}
\bibliographystyle{gost705s}
\bibliography{main}
% ----------------------------------------------------------------

\end{document}
