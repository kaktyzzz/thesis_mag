\section{Рекомендации по применению методов доступа}

При аутентификации с использованием сертификатов открытого ключа должны
использоваться аппаратные устройства аутентификации (смарт карты, USB ключи).
Краеугольным камнем аутентификации с открытым ключом является исключительное
владение пользователя закрытым ключом. Это не может быть обеспечено, если
закрытый ключ хранится или криптографические преобразования осуществляются на
компьютере пользователя.

Халатность пользователей типа 1. Устройства аутентификации могут оставляться на
рабочей станции. Если пользователь, отходя от своего рабочего места, оставляет
смарт карту в устройстве чтения или USB ключ в порту, кто-то другой в офисе
может легко представиться данным пользователем. Защита с помощью PIN кода
эффективна, если сеансы пользователей блокируются после определенного промежутка
бездействия, а для разблокирования необходима повторная аутентификация. Но PIN
коды можно узнавать, например, с помощью «подглядывания из-за плеча».
Организации должны призывать пользователей всегда носить устройства
аутентификации с собой. Это обеспечивается автоматически, если устройства
используются для физического контроля доступа в помещения (к примеру, смарт
карты или USB токены с интегрированными RFID метками)и операционная система
настроена на блокирование пользовательского сеанса при извлечении устройства
аутентификации.

Халатность пользователей типа 2. Устройства аутентификации могут быть утеряны.
Организация, в которой используются устройства аутентификации, будет вынуждена
каждому пользователю, потерявшему свое устройство, выдавать новое временное
устройство или предоставлять временную возможность аутентификации с помощью
альтернативного метода. При этом организация должна следить за тем, чтобы
подобные мероприятия не приводили к существенному ослаблению безопасности. В
качестве мер, уменьшающих риски при потере устройств аутентификации, можно
привести следующие правила издания сертификатов:
\begin{itemize}
  \item использование различных сертификатов (и соответственно ключевых пар) для целей
аутентификации, формирования ЭЦП и шифрования;
  \item использование генерации
ключевой пары вне устройства для сертификатов, предназначенных для шифрования
данных. Такой подход позволит иметь резервные копии ключевой информации на
случай утери устройства. Для архивирования закрытых ключей пользователей
рекомендуется использовать модули аппаратной защиты (HSM);
  \item использование
генерации ключевой пары с помощью устройства аутентификации для создания
сертификата, предназначенного для аутентификации пользователя в системе. Такой
подход позволяет гарантировать уникальность аутентификационных данных (то, что
существует только одно устройство, с помощью которого данный пользователь может
войти в систему, и что нельзя изготовить дубликат аутентификационных данных),
при этом при утере устройства (и последующих отзыве старого
сертификата и издании нового) не происходит потери пользовательской информации.
\end{itemize}

Устройства аутентификации могут быть уязвимы по отношению к логическим и
физическим атакам, а также атакам троянских коней. Каждая организация,
использующая данные устройства в приложениях безопасности, должна быть уверена в
том, что производители устройств и разработчики программного обеспечения
позаботились о принятии соответствующих логических и архитектурных контрмер.
Правда, эти меры могут отрицательно сказаться на удобстве использования
устройств аутентификации.

Логические атаки осуществляются, когда устройство аутентификации работает в
обычных физических условиях, но важная информация получается в результате чтения
трафика, входящего в устройство аутентификации и выходящего из него. Если есть
риск подобных атак, рекомендуется использовать устройства, позволяющие защищать
(к примеру, с помощью потокового шифрования) передаваемую информацию. В
компьютерных системах файлы данных, так или иначе, хранятся в пределах файловой
системы. Доступом к файлам и папкам управляет операционная система, и
рекомендации по назначению прав хорошо известны. Тем не менее список лиц,
которые могут получить доступ к незащищенным конфиденциальным данным,
оказывается весьма обширным. Администратор операционной системы по умолчанию
имеет самый высокий уровень прав доступа. Системный администратор имеет
возможность обратиться по сети к любому диску сервера с помощью так называемых
административных сетевых ресурсов. Не следует забывать и о роли человеческого
фактора. В случае ошибки администратора при настройке прав доступ к
конфиденциальным данным может получить любой пользователь. По оценкам IDC,
примерно 74\% финансовых потерь связаны с проблемами так называемого
«человеческого фактора». Для сравнения: потери от вирусных и хакерских атак
составляют соответственно 4 и 2\%.Также источником угроз являются компьютерные
вирусы, сетевые черви и троянские программы. Если список программного
обеспечения, устанавливаемого на сервере, всегда известен и само программное
обеспечение свободно от вирусов, то компьютеры пользователей могут быть заражены
вирусами и (или) содержать троянские программы. Используя параметры учетной
записи пользователя, работающего в данный момент за зараженным компьютером,
вирус может выполнять сканирование сетевых ресурсов и пытаться прочитать
хранящиеся на них файлы. Если таким пользователем является администратор, то
вирусу становится доступно содержимое дисков сервера через административные
сетевые ресурсы.

В качестве мер для предотвращения подобного рода угроз НСД к данным файловой
системы можно порекомендовать как организационные меры по контролю установленных
прав доступа и аудита их изменения, так и возможность использования
специализированных средств, таких, как программно аппаратные комплексы
криптографической защиты файловой системы с двухфакторной аутентификацией.

Подобные комплексы могут обеспечивать следующие функции:
\begin{itemize}
  \item двухфакторную аутентификацию для доступа к защищенным данным;
  \item защиту данных
на жестких и съемных дисках от несанкционированного доступа методом
«прозрачного» шифрования, в том числе от возможности получения доступа в обход
средств аутентификации и контроля доступа операционной системы с помощью
физического подключения к носителям данных;
  \item контроль сетевого доступа к
защищенным дискам, позволяющий для каждого защищенного диска определить, будет
ли он доступен по сети пользователям или только приложениям, выполняющимся
непосредственно на сервере;
  \item возможность отключения (в том числе с уничтожением
ключевой информации, с помощью которой проводится шифрование) доступа к
защищенной информации в случае возникновения нештатных ситуаций.
\end{itemize}

Примерами такого рода решений могут быть программно аппаратные комплексы для
защиты серверных данных AladdinSecretDisk Server NG, Физтехсофт StrongDisk
Server, ЛАН Крипто Индис Cервер, Control Break SafeBoot Device
Encryption,WinMagic SecureDoc. Доступ к ресурсам информационной системы -- это
фактически доступ к данным, находящимся в хранилище под управлением
СУБД. В таком контексте нарушением безопасности является НСД, который тем или
иным способом может получить потенциальный злоумышленник. Поэтому обеспечение
надежной и безопасной аутентификации — важнейшая задача как для разработчиков,
так и для администраторов систем. Существует несколько основных методов
аутентификации, которые обеспечивают разную степень безопасности. Выбор метода
должен основываться на таких факторах, как:
\begin{itemize}
  \item надежность, достоверность, неотказуемость (неотрицание совершения
  операций);
  \item стоимость решения, стоимость администрирования;
  \item соответствие общепринятым
стандартам;
  \item простота интеграции в готовые решения и новые разработки.
\end{itemize}

Как правило, более надежные методы являются более дорогостоящими как для
разработки, так и для администрирования.
 Наиболее надежным методом аутентификации на сегодняшний день является
 двухфакторная аутентификация с использованием цифровых сертификатов X.509,
 которые хранятся в защищенной памяти смарт карт или USB ключей. Оба этих
 устройства обеспечивают одинаковую степень надежности аутентификации, но смарт
 карты требуют дополнительного оборудования, что, естественно, увеличивает
 стоимость решения. Также необходимо обратить внимание на следующие аспекты:
 \begin{itemize}
   \item интеграцию с внешними системами аутентификации;
   \item безопасность соединений;
   \item защиту данных на любом уровне;
   \item выборочное шифрование данных;
   \item подробный аудит
действий пользователей;
   \item централизованное управление аутентификацией и
авторизацией пользователей.
 \end{itemize}

Обычно разработчики программного обеспечения не
уделяют должного внимания вопросам защиты данных, стараясь по максимуму
удовлетворить пожелания заказчиков по функциональности системы. В результате
подобные проблемы решаются зачастую в процессе внедрения системы, и это приводит
к ситуациям, когда «защищенное» приложение содержит <<секретные>> коды, прошитые
в исходных текстах программ или в файлах настройки. Определенные трудности для
разработчиков представляют ситуации, когда заказчик системы сам толком не
представляет, от кого и что, собственно, следует защищать. Интеграторы при
внедрении систем в организации заказчика уделяют большое внимание защите, но
здесь акценты смещены в сторону внешних атак -- троянов, вирусов, хакеров.
Вопросы <<внутренней>> безопасности и в этом случае остаются открытыми. В такой
ситуации усилия системных администраторов, неосведомленных в полной мере о
бизнес процессах организации, направлены на «урезание» прав пользователей, что
существенно осложняет работу системы в целом, а иногда приводит к ее полной
остановке. Мероприятия по защите информации должны включать решение следующих
вопросов:
\begin{itemize}
  \item определение угроз безопасности и групп потенциальных нарушителей;
  \item определение групп приложений, обеспечивающих бизнес процессы
  организации;
  \item определение категорий данных по уровню доступа;
  \item определение групп
пользователей системы;
  \item определение объектов, хранящих данные и подлежащих
защите;
  \item определение политик безопасности;
  \item определение методов хранения
связей пользователей;
  \item права доступа;
  \item данные;
  \item выбор метода аутентификации и
способов ее реализации;
  \item подготовка методики по обеспечению безопасности
инфраструктуры сети, СУБД, серверов приложений, клиентских приложений.
\end{itemize}

Также весьма популярным способом получения НСД к содержимому базы данных
является попытка прочитать файлы БД на уровне ОС (как обычные файлы). Если
администратором БД не были предприняты дополнительные меры по шифрованию
содержащейся в ней информации (а чаще всего это именно так), то простое
копирование файлов БД по сети и последующий прямой просмотр их содержимого
позволят злоумышленнику получить доступ к хранящейся в БД информации. Довольно
часто IT сотрудники не могут устоять от соблазна <<пронюхать>> информацию,
лежащую на поверхности. И они имеют все возможности для этого, поскольку обладают
привилегиями доступа к файлам. Именно поэтому защищенная система должна быть
спроектирована и построена так, чтобы:
\begin{itemize}
  \item файлы программного обеспечения СУБД не были доступны по сети;
  \item к файлам БД на
сервере не было доступа по сети;
  \item пользователи получают доступ к информации,
хранящейся в БД, только посредством СУБД;
  \item сами файлы БД были зашифрованы.
\end{itemize}

Шифрование файлов БД используется как простой и эффективный способ защиты данных
от несанкционированного доступа в случае физического обращения злоумышленника к
носителю информации -- сервер БД не выполнял функции файлового сервера.

Рекомендации к аутентификации сами по себе не могут рассматриваться как
полное решение проблемы аутентификации, управления учетными записями и правами
пользователей. Только используя их в различных комбинациях, можно создать
безопасное и удобное решение для централизованного управления
аутентификационными данными. К примеру, использование протокола Kerberos
совместно с LDAP каталогом (Light Weight Directory Access Protocol) может
рассматриваться как надежная платформа для построения.~\cite{auth_n_aut}