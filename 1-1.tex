\section{Корпоративные порталы}

Понятие корпоративного информационного портала (Enterprise Information Portal --
EIP) было введено компанией Merrill Lynch в ноябре 1998 года. Вслед за этим
последовало определение и многих других типов порталов.
Существует много самых разных определений корпоративным порталам, приведем
только исторически первое определение компании Merrill Lynch: <<Корпоративные
информационные порталы -- это приложения, которые позволяют компаниям раскрывать
информацию, хранящуюся внутри и вне организации, и предоставлять пользователям
единый шлюз доступа к персонализированной информации, необходимой для принятия
бизнес-решений>>. Они являются <<…совокупностью программных приложений, которые
консолидируют, управляют, анализируют и распространяют информацию по всему
предприятию и за его пределами>>~\cite{kis_eip}.

Определение корпоративного портала должно исходить из той цели, которую компания
ставит (или может ставить, но еще не знает об этом) перед корпоративным
порталом. В этом смысле у каждой компании будет свое определение, но именно того
корпоративного портала, который ей и требуется.

Если отойти от существующих определений и от вопроса возможностей корпоративного
портала, то для территориально разделенной компании имеющей корпоративную
информационную систему (КИС) в центральном офисе, можно сформулировать следующие
цели создаваемого корпоративного портала:
\begin{enumerate}
  \item Обеспечить всем работникам компании независимо от их территориального
нахождения доступ к КИС компании (естественно, в соответствии с полномочиями
разграничения доступа), доступ и к требуемой информации, и к требуемым
приложениям, необходимым для плодотворной работы. Т. е. при такой постановке
корпоративных информационных порталов (далее КИП) -- это средство обеспечения
прозрачного доступа к центральной КИС для всех сотрудников;
\item Обеспечить доступ к КИС компании партнерам и клиентам компании (естественно,
в соответствии с самыми строгими полномочиями разграничения доступа), доступ и к
требуемой информации, и к требуемым приложениям, необходимым для получения
информации, осуществления заказа и т.д.
\end{enumerate}

При этом корпоративный портал, фактически, является веб-интерфейсом КИС. Многие
производители КИС класса ERP включают в свои системы корпоративные порталы и
веб-интерфейсы (Oracle, Baan, Парус), а совсем недавно появились КИС класса
ERP II, основные отличия которых от ERP-КИС следующие:
\begin{enumerate}
  \item Выход КИС за пределы компании, <<прозрачное>> взаимодействие компании с
контрагентами (заказчиками, поставщиками, банками, налоговыми органами и др.).
Открытие максимального числа информационных процессов, развитие так называемой
<<коллаборативной коммерции>> (c-commerce) -- совместного электронного бизнеса
предприятия, деловых партнеров и потребителей;
\item Максимально полный охват всех бизнес-процессов компании.
\end{enumerate}

Таким образом, понятия корпоративной информационной системы и корпоративного
портала начинают сливаться. Поэтому можно сформулировать наиболее полное
определение корпоративных информационных порталов.

Корпоративные информационные порталы --- это web-сайты, которые позволяют
компании раскрывать информацию, хранящуюся внутри и вне организации, и предоставить
каждому пользователю единую точку доступа к предназначенной для него информации,
необходимой для принятия обоснованных управленческих решений.~\cite{opred_kis}

Исходя из функциональных особенностей сети корпоративных порталов, одну из
основополагающих ролей играет система аутентификации пользователей и
разграничение прав доступа.

Для определения принадлежности пользователя к сети корпоративных порталов
необходимо осуществить эффективный алгоритм аутентификации. Основные методы
аутентификации пользователей различаются, прежде всего, аутентификациионными
факторами. Аутентификационный фактор --- определенный вид информации,
предоставляемый субъектом системе при его аутентификации -- нечто, нам
известное: пароль. Отличительная характеристика представляет собой секретную
информацию, которая неизвестна непосвященным людям. При некомпьютерном использовании это
может быть произносимый голосом пароль или запоминаемая комбинация для замка. В
случае вычислительных систем это может быть пароль, вводимый с помощью
клавиатуры. Разработчики могут дешево и легко реализовать паролевый механизм.
Запоминаемое секретное слово может быть наиболее удобным средством с точки
зрения перемещающихся пользователей, т.е. для людей, которые подключаются к
системе из непредсказуемых удаленных мест. Однако использование паролей не
лишено недостатков. Во-первых, их эффективность зависит от секретности, а
хранить пароли в тайне нелегко. Существует бессчетное количество способов
выведать или перехватить пароль, и обычно обнаружить активную разведку до
нанесения урона невозможно. Во-вторых, развитие методов нападения сделало для
взломщиков определение паролей, обычно выбираемых людьми, относительно простым
делом. Даже если выбираются трудно угадываемые пароли, их где-нибудь записывают,
чтобы при необходимости иметь под рукой. Но, конечно же, записываемый пароль
более уязвим в плане возможной кражи, чем запоминаемый. Даже действующие из
лучших побуждений люди в какой-то момент нарушают правила использования паролей
просто для того, чтобы иметь возможность воспользоваться своим компьютером
именно тогда, когда в этом возникнет необходимость. Нечто, присущее нам:
биометрика. Отличительной характеристикой является какая-нибудь физическая
особенность, уникальная для аутентифицируемого лица. До появления компьютеров
это могли быть личная подпись, фотография, отпечаток пальца или письменное
описание внешнего вида человека.~\cite{auth_n_aut}

При компьютерном использовании отличительная характеристика физического лица
измеряется и сравнивается с ранее полученными данными, снятыми с достоверно
установленной личности. В хорошо известных методиках для аутентификации
используются голос человека, отпечатки пальцев, письменная подпись, форма кисти
или особенности радужной оболочки глаза. Нечто, имеющееся у нас: устройство
аутентификации. Отличительной характеристикой является наличие у авторизованного
лица некоего конкретного предмета. При некомпьютерном использовании это могли
быть печать или ключ от замка. В компьютерных системах это может быть не более
чем носитель с файлом данных, содержащим отличительную характеристику. Часто
характеристика встраивается в устройство, например в карту с магнитной полосой,
смарт-карту, USB ключ или в OTP токен. Кроме того, подобные устройства часто
предлагают возможность так называемой многофакторной аутентификации.
Многофакторная аутентификация — аутентификация, в процессе которой используются
аутентификационные факторы нескольких типов. Например, пользователь должен
предоставить смарт-карту и дополнительно ввести пароль. Также используются
понятия двухфакторной и трехфакторной аутентификации при использовании в
процессе аутентификации комбинации двух и трех типов аутентификационных факторов
соответственно.

Метод аутентификации (метод регистрации) --- специфика использования
определенного типа аутентификационных факторов в процедуре аутентификации. Кроме
методов локальной аутентификации -- входа пользователя локально в компьютер,
существуют методы для аутентификации в локальных вычислительных сетях, при
доступе к удаленным ресурсам, веб приложениям и др.

Немаловажной составляющей корпоративных информационных порталов является
электронный документооборот. Документооборот --- движение документов в
организации с момента их создания или получения до завершения исполнения или
отправления.~\cite{GOST_P51141}

Комплекс работ с документами: прием, регистрация, рассылка, контроль исполнения,
формирование дел, хранение и повторное использование документации, справочная
работа.

Электронный документооборот (ЭДО) представляет собой единый механизм по работе с
документами, представленными в электронном виде, с реализацией концепции
<<безбумажного делопроизводства>>.

Необходимость в автоматизации управления документооборотом обусловлено
повышением эффективности организационно-распорядительного документооборота и
работы функциональных специалистов, создающих документы и использующих их в
повседневной работе.

Достоинства электронного документооборота:
\begin{enumerate}
  \item многокритериальный поиск документов;
  \item контроль исполнения документов;
  \item регистрация документов;
  \item ввод резолюций к документам;
  \item распределенная
обработка документов в сети;
  \item  распределение прав доступа к различным
документам и функциям системы;
  \item ведение нескольких картотек документов;
  \item работа с проектами документов;
  \item распределение находящихся на исполнении
документов по <<папкам>> в зависимости от стадии исполнения документа:
поступившие, на исполнении, на контроле и другие;
  \item формирование стандартных
отчетов;
  \item обмен документами по электронной почте;
  \item списание документов в дело;
  \item отслеживание перемещений бумажных оригиналов и копий документов,
ведение реестров внутренней передачи документов;
  \item ведение пользовательских
списков должностных лиц, организаций, тематических рубрик, групп документов;
  \item редактирование шаблонов выходных печатных форм.~\cite{kuznecov}
\end{enumerate}

Введение электронного документооборота позволяет снизить количество служб,
занятых работой с документами (курьеров, канцелярских работников и т. п.).
Для обеспечения электронного документооборота необходимо использование
подсистемы управления обменом электронно-цифровой подписи.

Электронно-цифровая подпись --- атрибут электронного документа, используемый для
защиты информации от несанкционированного использования и подделки.

Электронно-цифровая подпись формируется путем криптографического преобразования
информации с закрытым ключом, что позволяет определить владельца сертификата
ключа подписи и обеспечить неотказуемость подписавшегося от документа, а также
проверить полученную информацию на отсутствие ошибок и неточностей. Электронный
документ --- это документ, подготовленный с использованием системы электронного
документооборота (далее СЭД), зафиксированный на материальном носителе в виде
объекта СЭД и снабженный реквизитами, с помощью которых можно идентифицировать место, время
создания и автора документа.

 Чаще всего цифровые подписи используются для подтверждения имени отправителя,
 основываясь на том, что лишь он один владеет уникальным закрытым ключом,
 которому соответствует полученный открытый ключ. Также электронная подпись
 иногда используется для датирования документа с помощью штампа времени:
 доверенная сторона подписывает документ штампом времени с помощью своего
 специального закрытого ключа, подтверждая существование документа в данный
 момент, обозначенный в цифровой подписи.
 
В случае если электронно-цифровая подпись ставится для удостоверения личности
отправителя: сторона, которой доверяют априори, предоставляет открытый ключ и
информацию о владельце закрытого ключа получателю документа -- получается
иерархия доверия.
