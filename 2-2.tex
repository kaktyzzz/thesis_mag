\section{Уязвимость операционных систем} 

Внутренняя структура современных ОС чрезвычайно сложна, поэтому проводить
адекватную политику безопасности и защищать ее гораздо труднее, чем в случае
СУБД. Это обусловлено большим числом различных типов защищаемых объектов и
информационных потоков в современных ОС. Операционная система имеет сложную
внутреннюю структуру и поэтому задача построения адекватной политики
безопасности для ОС решается сложнее, чем для СУБД.
Наилучшие результаты атак достигаются при использовании самых простых методов
взлома через выявленные лазейки в защите ОС -- чем проще алгоритм атаки, тем
больше вероятность того, что атака пройдет успешно. Возможность практической
реализации той или иной атаки на ОС в значительной мере определяется
архитектурой и конфигурацией ОС. Но есть атаки, которые могут быть применены
практически к любой ОС.~\cite{zapechnikov}

\begin{enumerate}
  \item Кража пароля:
  \begin{itemize}
    \item подглядывание за легальным пользователем, когда тот вводит пароль
    (даже если во время ввода пароль не высвечивается на экране, его можно легко
    узнать, следя за перемещением пальцев пользователя по клавиатуре);
	\item получение пароля из файла, в котором он был сохранен <<ленивым>>
	пользователем, не желающим каждый раз затруднять себя вводом пароля при сетевом
	подключении (как правило, такой пароль хранится в незашифрованном виде);
	\item  поиск пароля, записанного на календаре, в записной книжке или на
	оборотной стороне компьютерной клавиатуры (особенно часто подобная ситуация
	встречается, когда администратор заставляет пользователей применять длинные,
	трудно запоминаемые пароли);
	\item кража внешнего носителя парольной информации (дискеты или электронного
	ключа, на которых хранится пароль пользователя для входа в ОС); перехват пароля
	программной закладкой.
  \end{itemize}	
  \item Подбор пароля:
  \begin{itemize}
    \item полный перебор всех возможных вариантов пароля (метод <<грубой
    силы>>);
	\item оптимизированный перебор вариантов пароля: по частоте встречаемости
	символов, с помощью словарей наиболее часто встречающихся паролей, с
	привлечением знаний о конкретном пользователе, с использованием сведений о
	существовании эквивалентных паролей -- тогда из каждого класса эквивалентности
	опробуется всего один пароль, что значительно сокращает время перебора.
  \end{itemize}
  \item Сканирование <<жестких>> дисков компьютера: 
  \begin{itemize}
    \item злоумышленник последовательно
пытается обратиться к каждому файлу, хранимому на <<жестких>> дисках
пользователей сети (если объем дискового пространства достаточно велик, можно
быть вполне уверенным, что при описании доступа к файлам и каталогам
администратор допустил хотя бы одну ошибку, в результате чего все такие каталоги
и файлы будут прочитаны взломщиком);
    \item чтобы скрыть следы, злоумышленник может выступать под чужим именем --
  например, под именем легального пользователя, чей пароль ему известен.    
  \end{itemize}
  \item Сборка <<мусора>> с дисков компьютера и в оперативной памяти: 
  если средства ОС позволяют восстанавливать ранее удаленные объекты,
  злоумышленник может получить доступ к объектам, удаленным другими
  пользователями, просмотрев содержимое их <<мусорных корзин>>.
  \item Превышение полномочий, т.е. используя ошибки в ПО или в
  администрировании ОС, злоумышленник получает полномочия, превышающие те,
  которые предоставлены ему согласно действующей политики безопасности:
  \begin{itemize}
    \item запуск программы от имени пользователя, имеющего необходимые
    полномочия, или в качестве системной программы (драйвера, сервиса, демона и
    т. д.), выполняющейся от имени ОС;
	\item подмена динамически загружаемой библиотеки, используемой системными
	программами, или изменение переменных среды, описывающих путь к таким
	библиотекам; 
	\item модификация кода или данных подсистемы защиты ОС.
  \end{itemize}

  \item Отказ в обслуживании (целью этой атаки является частичный или полный
  вывод ОС из строя):
  \begin{itemize}
    \item захват ресурсов, т.е. программа злоумышленника производит захват всех
    имеющихся в ОС ресурсов, а затем входит в бесконечный цикл;
	\item бомбардировка запросами -- программа злоумышленника постоянно направляет
	ОС запросы, реакция на которые требует привлечения значительных ресурсов сети;
	\item использование ошибок в ПО или администрировании.~\cite{zapechnikov}
  \end{itemize}

\end{enumerate}