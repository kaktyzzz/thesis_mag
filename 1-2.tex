\section{Анализ технологических решений по осуществлению аутентификации}

Для осуществления аутентификации существует множество технологических решений.
Рассмотрев все возможные технические и программные аспекты, можно будет сделать
вывод относительно принятия решения о применении наиболее эффективного
алгоритма.

\subsection{Биометрический метод}

Современные технологии способны обеспечить удостоверение личности человека,
используя характерные только для него одного характеристики. Данные технологии
основаны на практическом применении знаний научной дисциплины биометрии. Данная
дисциплина занимается статистическим анализом биологических наблюдений и
явлений. Биометрическая характеристика --- это измеримая физиологическая или
поведенческая черта живого человека. Некоторые биометрические характеристики
уникальны для данного человека, и их можно использовать для установления
личности или проверки декларируемых личных данных:
\begin{itemize}
  \item для идентификации пользователя (вместо ввода имени пользователя);
  \item для однофакторной аутентификации пользователя;
  \item совместно с паролем или
аутентификационным токеном(таким, как смарт-карта);
  \item для обеспечения
двухфакторной аутентификации.
\end{itemize}

Биометрические характеристики делятся на группы. Физиологические биометрические
характеристики (также называемые физическими биометрическими характеристиками,
статическими биометрическими характеристиками) --- биометрические
характеристики, основанные на данных, полученных путем измерения анатомических
характеристик человека, таких, как отпечаток пальца, форма лица или кисти,
сетчатка глаза.

Поведенческие биометрические характеристики (также называемые динамическими
биометрическими характеристиками) -- биометрические характеристики, основанные
на данных, полученных путем измерения действий человека.
Характерной чертой для поведенческих характеристик является их протяженность во
времени -- измеряемое действие имеет начало, середину и конец. К примеру: голос,
подпись. Хотя биометрические технологии отличаются в объектах и способах
измерений, все биометрические системы работают одинаково. Пользователь
предоставляет образец (sample) -- опознаваемое необработанное изображение или
запись физиологической или поведенческой характеристики посредством
регистрирующего устройства (например, сканера или камеры). Этот биометрический
образец обрабатывается для получения информации об отличительных признаках, в
результате чего получается контрольный шаблон (или шаблон для проверки). Шаблоны
представляют собой числовые последовательности. Сам образец невозможно
восстановить из шаблона. Результат проверки других аутентификационных данных,
как правило, однозначен --- это решение «да» или «нет» (пароль совпал или нет).

В случае проверки контрольного шаблона результат иной. Контрольный шаблон
сравнивается с эталонным шаблоном (или зарегистрированным шаблоном), созданным
на основе нескольких образцов, определенной физиологической или поведенческой
характеристики пользователя, взятых при его регистрации в биометрической
системе. Поскольку эти два параметра (контрольный и эталонный шаблон) полностью
никогда не совпадают, то степень совпадения должна превышать определенную
настраиваемую пороговую величину. Соответственно в биометрических системах
контрольный шаблон может быть ошибочно признан:
\begin{itemize}
  \item соответствующим эталонному шаблону другого лица;
  \item не соответствующим эталонному шаблону данного
пользователя, несмотря на то, что этот пользователь зарегистрирован в
биометрической системе.
\end{itemize}

Точность биометрической системы измеряется двумя
параметрами:

\begin{itemize}
  \item коэффициентом неверных совпадений (FMR), также известным под названием «ошибка
типа I» или «вероятность ложного допуска» (FAR);
  \item коэффициентом неверных
несовпадений (FNMR), также известным под названием «ошибка типа II» или
«вероятность ложного отказа в доступе» (FRR).
\end{itemize}~\cite{auth_n_aut} 

Биометрическая аутентификация обычно является одним из наиболее легких подходов
для тех людей, которые должны проходить аутентификацию. В большинстве случаев
хорошо спроектированная биометрическая система просто снимает показания с
человека и правильно выполняет аутентификацию. Однако все преимущества сводятся
на нет несколькими недостатками. Как правило, по сравнению с другими системами,
приобретение, установка и эксплуатация оборудования стоят дорого. При
дистанционном использовании биометрические показания подвержены риску перехвата:
похититель может воспроизвести запись показаний, чтобы замаскировать себя под
владельца, или использовать их в целях выслеживания последнего. Если
биометрические показатели попадают в плохие руки, то их владелец не имеет
способа восполнения ущерба, так как биометрические особенности невозможно
изменить. Кроме того, сам процесс аутентификации сложен. Сложно также сделать
систему достаточно чувствительной, чтобы она отвергала посторонних пользователей
и при этом время от времени не отвергала своих. Биометрические показатели также
могут быть признаны негодными вследствие физиологических изменений и телесных
повреждений. Был случай, когда биометрическое устройство не впустило в режимное
помещение работавшую там женщину, потому что из-за беременности у нее изменилась
картина кровеносных сосудов в сетчатке глаз. Несмотря на то, что криптография с
открытым ключом может обеспечивать аутентификацию пользователя, сам по себе
закрытый ключ подобен паспорту без фотографии. Закрытый ключ, хранящийся на
жестком диске компьютера владельца, уязвим по отношению к прямым и сетевым
атакам. Достаточно подготовленный злоумышленник может похитить персональный ключ
пользователя и с помощью этого ключа представляться этим пользователем. Защита
ключа с помощью пароля помогает, но недостаточно эффективно -- пароли уязвимы по
отношению ко многим атакам. Несомненно, требуется более безопасное хранилище.

\subsection{Смарт-карты}

Смарт-карты --- пластиковые карты стандартного размера банковской карты
(стандарт ISO 7816 1), имеющие встроенную микросхему. Они находят все более широкое
применение в различных областях, от систем накопительных скидок до кредитных и
дебетовых карт, студенческих билетов и телефонов стандарта GSM. Для
использования смарт-карт в компьютерных системах необходимо устройство чтения
смарт-карт. Несмотря на название -- устройство чтения (или считыватель),
большинство подобных оконечных устройств, или устройств сопряжения (IFD),
способны как считывать, так и записывать информацию, если позволяют возможности
смарт-карты и права доступа. Устройства чтения смарт-карт могут подключаться к
компьютеру посредством последовательного порта, слота PCMCIA или USB. Устройство
чтения смарт-карт также может быть встроено в клавиатуру. Как правило, для
доступа к защищенной информации, хранящейся в памяти смарт-карты, требуется
пароль, называемый PIN кодом (англ. Personal Identification Number --
персональный идентификационный номер).~\cite{thesis_smart_card}

\subsection{Аппаратный генератор одноразовых паролей}

В основе всех методов аутентификации с использованием пароля лежит предположение
о том, что только законный пользователь может успешно пройти проверку личности,
что только он знает свой пароль. Часто злоумышленник может легко узнать
идентификатор пользователя. Особенно если в качестве идентификатора пользователя
используется не случайная строка символов, состоящая из букв и цифр, а «имя
пользователя». Так как обычно <<имя пользователя>> -- это различные варианты
комбинации имени и фамилии пользователя, то определить имя пользователя не
составляет большого труда. Соответственно, если злоумышленнику удастся узнать
еще и пароль, то злоумышленнику будет легко представиться соответствующим
пользователем. Сколь бы ни был пароль засекречен, узнать его иногда не слишком
трудно. Злоумышленник может сделать это, используя различные способы атак. Для
каждой из этих атак есть методы защиты. Но большинство из этих методов обладают
различными недостатками. Некоторые виды защиты достаточно дороги (борьба с
паразитным излучением оборудования), другие создают неудобства для пользователей
(правила формирования пароля, использование длинных паролей).

Один из вариантов защиты от различных атак на аутентификацию по паролю -- это
переход на аутентификацию с использованием одноразовых паролей.

Одноразовые пароли (One Time Passwords -- OTP) -- динамическая
аутентификационная информация, генерируемая для единичного использования с помощью
аутентификационных токенов (программных или аппаратных).
OTP токен -- мобильное персональное устройство, принадлежащее определенному
пользователю, генерирующее одноразовые пароли, используемые для аутентификации
данного пользователя. Аутентификация с одноразовым паролем обладает
устойчивостью к атаке анализа сетевых пакетов, что дает ей значительное
преимущество перед запоминаемыми паролями. Несмотря на то, что злоумышленник
может перехватить пароль методом анализа сетевого трафика, поскольку пароль
действителен лишь один раз или в течение ограниченного промежутка времени, у
злоумышленника, в лучшем случае, есть весьма ограниченная возможность
представиться пользователем посредством перехваченной информации. Для того чтобы
сгенерировать OTP, необходимо иметь OTPтокен. Таким образом, при использовании
OTP вместо аутентификационного фактора <<нечто, нам известное>>, применяется
другой аутентификационный фактор -- <<нечто, имеющееся у нас>>.

Другим важным преимуществом аутентификационных устройств является то, что многие
из них требуют от пользователя введения PIN кода или пароля, который может
использоваться различными способами, в частности:
\begin{itemize}
  \item для активации OTP токена;
  \item в качестве дополнительной информации, используемой
при генерации OTP;
  \item для предъявления серверу аутентификации вместе с OTP.
\end{itemize}

В этих методах аутентификации используются два аутентификационных фактора.
Поэтому они относятся к двухфакторной аутентификации. OTP токены имеют небольшой
размер и выпускаются в виде(форм факторы):
\begin{itemize}
  \item карманного калькулятора;
  \item брелока;
  \item смарт карты;
  \item устройства,
комбинированного с USB ключом;
  \item специального программного обеспечения для
карманных компьютеров.
\end{itemize}

В качестве примера решений OTP можно привести линейкуRSA
SecurID, ActivCard Token, комбинированный USB ключAladdin eToken NG OTP.
~\cite{auth_n_aut}

\subsection{USB ключи}

Некоторые производители выпускают другие виды аппаратных устройств,
представляющие собой комбинацию смарт карты и устройства чтения смарт карт. Они
по свойствам памяти и вычислительным возможностям полностью аналогичны смарт
картам. Наиболее популярны аппаратные «ключи», использующие порт USB. USB ключи
привлекательны для некоторых организаций, поскольку USB стал стандартным портом
для подключения периферийных устройств: организации не нужно приобретать для
пользователей какие бы то ни было считыватели. Аутентификацию на основе смарт
карт и USB ключей сложнее всего обойти, так как используется уникальный
физический объект, которым должен обладать человек, чтобы войти в систему. В
отличие от паролей владелец быстро узнает о краже и может сразу принять
необходимые меры для предотвращения ее негативных последствий. Кроме того,
реализуется двухфакторная (а в случае решений совместно с биометрической
информацией и трехфакторная) аутентификация. Основными слабыми местами являются
более высокая стоимость реализации и риск дополнительных расходов, связанных с
потерей аппаратуры. Смарт карты, USB ключи и другие устройства аутентификации
могут повысить надежность служб PKI: смарт карта может использоваться для
безопасного хранения закрытых ключей пользователя, а также для безопасного
выполнения криптографических преобразований. Безусловно, устройства
аутентификации не обеспечивают абсолютную безопасность, но надежность их защиты
намного превосходит возможности обычного настольного компьютера. Хранить и
использовать закрытый ключ можно по-разному, и разные разработчики используют
различные подходы.

Наиболее простой из них -- использование устройства аутентификации в качестве
защищенного носителя аутентификационной информации: при необходимости карта
экспортирует закрытый ключ, и криптографические операции осуществляются на
рабочей станции. Этот подход является не самым совершенным с точки зрения
безопасности, зато относительно легко реализуемым и предъявляющим невысокие
требования к устройству аутентификации. В качестве примера подобного рода
устройств аутентификации можно привести Aladdin eToken R2, Rainbow iKey 1000,
Актив ruToken. Два следующих подхода более безопасны, поскольку предполагают
выполнение устройством аутентификации криптографических операций. В первом
случае пользователь генерирует ключи на рабочей станции и сохраняет их в памяти
устройства, во втором — с помощью устройства. В обоих случаях, после того как
закрытый ключ сохранен, его нельзя извлечь из устройства и получить любым другим
способом. Генерация ключевой пары вне устройства. В этом случае пользователь
может сделать резервную копию закрытого ключа. Если устройство выйдет из строя,
будет потеряно, повреждено или уничтожено, пользователь сможет сохранить тот же
закрытый ключ в памяти нового устройства. Это необходимо, если пользователю
требуется расшифровать какие-либо данные, сообщения и так далее, зашифрованные с
помощью соответствующего открытого ключа. Однако при этом закрытый ключ
пользователя подвергается риску быть похищенным.Генерация ключевой пары с
помощью устройства. В этом случае закрытый ключ не появляется в открытом виде и
нет риска, что злоумышленник украдет его резервную копию. Единственный способ
использования закрытого ключа -- это обладание устройством аутентификации.
Являясь наиболее безопасным, это решение выдвигает высокие требования к
возможностям самого устройства: оно должно обладать функциональностью генерации
ключей и осуществления криптографических преобразований. Это решение также
предполагает, что закрытый ключ не может быть восстановлен в случае выхода
устройства из строя и т.п. Об этом необходимо беспокоиться при использовании
закрытого ключа для шифрования, но не там, где он используется для
аутентификации, или в других службах, использующих цифровые подписи. Подобным
образом способны работать процессорные смарткарты и USB токены на их основе, к
примеру Aladdin eTokenPRO, eToken NG OTP, Rainbow iKey 2000, iKey 3000,
AthenaASECard Crypto, Schlumberger Cryptoflex, ActivCard USB Key и др.~\cite{auth_n_aut}