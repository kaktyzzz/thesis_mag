\section{Уязвимость архитектуры клиент-сервер} 

В основе общения по открытым сетям лежит технология клиент-сервер. Определений
этой архитектуры очень много. В общем случае это такой способ проектирования ИС,
при котором она может быть рассмотрена как совокупность некоторого числа систем
двух видов -- клиентской и серверной.
Как уже отмечалось выше, клиентская часть системы инициирует запросы, а
серверная обрабатывает запросы и при необходимости генерирует ответы клиенту. В
общем случае серверная часть состоит из нескольких элементов --- ОС, СУБД,
прикладной системы, в которой реализована общая бизнес-логика для всех клиентов.
ОС предоставляет необходимые сервисные возможности и программные интерфейсы API
для СУБД, которая в своей БД обрабатывает и выполняет запросы прикладной
системы.~\cite{ComputerWeek}

Угрозы в сетевой среде можно разделить на следующие виды:
\begin{itemize}
  \item прослушивание сети;
  \item изменение корпоративных потоков данных;
  \item воздействие на инфраструктурные сетевые сервисы;
  \item подделка сетевых пакетов;
  \item генерация и посылка аномального трафика (пакетов);
  \item отказ от совершенных действий.
\end{itemize}
	
Прослушивание сети может предприниматься злоумышленниками для достижения следующих целей:
\begin{itemize}
  \item перехвата пересылаемых сведений;
  \item перехвата аутентификационной информации;
  \item анализа трафика.
\end{itemize}
	
Изменение корпоративных потоков данных влечет за собой следующие нарушения
безопасности:
\begin{itemize}
  \item кражу, переупорядочение, дублирование информации;
  \item изменение и вставку собственных данных (нелегальный посредник).
\end{itemize}
	
Воздействие на инфраструктурные сетевые сервисы означает:
\begin{itemize}
  \item вмешательство в работу сервиса имен;
  \item изменение маршрутов корпоративных потоков информации.
\end{itemize}
	
Подделка сетевых пакетов может принимать следующие формы:
\begin{itemize}
  \item подделка адресов;
  \item перехват соединений;
  \item имитация работы других серверов.
\end{itemize}
	
Генерация и посылка аномальных пакетов представляют собой атаки на доступность,
получившие в последнее время относительно широкое распространение.
Наконец, отказ от совершенных действий --- это угроза прикладного уровня, она
реальна в первую очередь в силу распределенности систем клиент-сервер.
~\cite{ComputerWeek} Список наиболее очевидных угроз в архитектуре клиент-сервер выглядит
следующим образом:
\begin{itemize}
  \item пассивный перехват передаваемых запросов;
  \item модификация (активный перехват) передаваемых запросов;    
  \item пассивный перехват ответов клиенту;
  \item модификация ответов клиенту;
  \item выдача злоумышленником себя за определенный сервер;
  \item выдача злоумышленником себя за определенного клиента;
  \item перегрузка сервера выдачей большого числа случайных запросов, что может
привести к отказу обслуживания новых клиентов; 
  \item случайные сбои и ошибки функционирования аппаратуры и программных
  элементов сервера; 
  \item злоумышленные действия зарегистрированных клиентов;
  \item другие виды атак на ПО сервера.
\end{itemize}